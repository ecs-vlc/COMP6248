\documentclass{beamer}
\usepackage{lmodern}
\usepackage{listings}
\usepackage{amsmath}
\usepackage{bm}
\usepackage{textpos} % package for the positioning

\usepackage{pgf, tikz}
\usetikzlibrary{arrows, automata}

\usetheme{Copenhagen}
\hypersetup{pdfstartview={Fit}}
\lstset{basicstyle=\small\ttfamily,breaklines=true}

\title[COMP6248 Deep Learning]{COMP6248 Differentiable Programming}
\subtitle{(and some Deep Learning)}
\author{Jonathon Hare and Kate Farrahi}
\institute[]
{
  Vision, Learning and Control\\
  University of Southampton 
}
\date{}
\subject{Computer Science}
\useoutertheme{infolines}
\setbeamertemplate{headline}{} %remove headline
\setbeamertemplate{navigation symbols}{} %remove navigation symbols

\begin{document}
  \frame{
  \titlepage
}

\begin{frame}{pause}
\frametitle{Machine Learning - A Recap}
{\tiny All credit for this slide goes to Niranjan}\\
\vspace{5mm}
\begin{tabular}{ll}
Data & $\{\bm{x}_n, \bm{y}_n\}^N_{n=1} \qquad \{\bm{x}_n\}^N_{n=1}$ 
\vspace{3mm} \\ \pause
Function Approximator & $\bm{y} = f (\bm{x}, \bm{\theta}) + \nu$ 
\vspace{3mm} \\ \pause
Parameter Estimation & $E_0 = \sum^N_{n=1} \{\|\bm{y}_n - f (\bm{x}_n; \bm{\theta})\|\}^2$
\vspace{3mm} \\ \pause
Prediction & $\bm{\hat y}_{N+1} = f(\bm{x}_{N+1}, \bm{\hat \theta})$
\vspace{3mm} \\ \pause
Regularization & $E_1 = \sum^N_{n=1} \{\|\bm{y}_n - f (\bm{x}_n; \bm{\theta})\|\}^2 + g(\|\bm\theta\|)$
\vspace{3mm} \\ \pause
Modelling Uncertainty & $p(\bm\theta|\{\bm x_n, \bm y_n\}_{n=1}^N)$
\vspace{3mm} \\ \pause
Probabilistic Inference & $\mathop{\mathbb{E}}[g(\bm\theta)] = \int g(\bm\theta)p(\bm\theta)d\bm\theta = \frac{1}{N_s}\sum_{n=1}^{N_s}g(\bm\theta^{(n)})$
\vspace{3mm} \\ \pause
Sequence Modelling & $\bm x_n = f(\bm x_{n-1}, \bm\theta)$
\end{tabular}
\vspace{5mm}
\end{frame}

\begin{frame}
\frametitle{What is Deep Learning?}

Deep learning is primarily characterised by function compositions: \\ \vspace{10mm}
\begin{itemize}
	\item<+-> Feedforward networks: $\bm{y} = f (g(\bm{x}, \bm\theta_g), \bm{\theta_f})$
	\begin{itemize}
		\item Often with relatively simple functions (e.g. $f(\bm x, \bm{\theta}_f) = \sigma(\bm{x}^\top \bm{\theta}_f)$)
	\end{itemize} \vspace{3mm}
	\item<+-> Recurrent networks: $\bm y_t = f(\bm y_{t-1}, \bm x_t, \bm\theta) = f(f(\bm y_{t-2}, \bm x_{t-1}, \bm\theta), \bm\theta) = \dots$
\end{itemize}
\vspace{10mm}

\uncover<+->{
In the early days the focus of deep learning was on learning functions for classification. Nowadays the functions are much more general in their inputs and outputs.
}

\end{frame}

\begin{frame}
\frametitle{What is Differentiable Programming?}
	
\begin{itemize}
	\item<+-> Differentiable programming is a term coined by Yann Lecun\footnote{https://www.facebook.com/yann.lecun/posts/10155003011462143} to describe a superset of Deep Learning.
	\item<+-> Captures the idea that computer programs can be constructed of parameterised functional blocks in which the parameters are learned using some form of gradient-based optimisation.
	\begin{itemize}
		\item<+-> The implication is that we need to be able to compute gradients with respect to the parameters of these functional blocks. We'll start explore this in detail next week...
		\item<+-> The idea of Differentiable Programming also opens up interesting possibilities: 
		\begin{itemize}
			\item The functional blocks don't need to be direct functions in a mathematical sense; more generally they can be \emph{algorithms}.
			\item What if the functional block we're learning parameters for is itself an algorithm that optimises the parameters of an internal algorithm using a gradient based optimiser?!\footnote{See our ICLR 2019 paper: https://arxiv.org/abs/1812.03928}
		\end{itemize}
	\end{itemize}
\end{itemize}
\end{frame}

\begin{frame}
\frametitle{Is all Deep Learning Differentiable Programming?}
\begin{itemize}
	\item Not necessarily!
	\begin{itemize}
		\item<+-> Most deep learning systems are trained using first order gradient-based optimisers, but there is an active body of research on gradient-free methods.
		\item<+-> There is an increasing interest in methods that use different styles of learning, such as Hebbian learning, within deep networks. More broadly there are a number of us\footnote{including at least myself, my PhD students and Geoff Hinton!} who are interested in biologically motivated models and learning methods.
	\end{itemize}
\end{itemize}
\end{frame}

\begin{frame}
	\frametitle{Why should we care about this?}
	
\end{frame}

\begin{frame}
	\frametitle{Where did it all start \& what was the motivation?}
	
\end{frame}

\begin{frame}
	\frametitle{What is the objective of this module?}
	
\end{frame}


\begin{frame}
	\frametitle{What will we cover in the module?}
	
\end{frame}

\begin{frame}
	\frametitle{How is this module going to be delivered?}
	
\end{frame}

\begin{frame}
	\frametitle{Lecture \& lab session plan}
	
\end{frame}

\begin{frame}
	\frametitle{What do we expect you already know?}
	
\end{frame}

\begin{frame}
	\frametitle{What might you already know?}
	
\end{frame}

\begin{frame}
	\frametitle{Assessment Structure}
	
\end{frame}

\begin{frame}
	\frametitle{Assessment Timetable}
	
\end{frame}

\begin{frame}
	\frametitle{The Main Assignment}
	\framesubtitle{The ICLR Reproducibility Challenge}
	
\end{frame}

\end{document}